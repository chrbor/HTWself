


%		****************************************************************************
%			Übersicht zu den Funktionsbefehlen für die Eintragungen
%		****************************************************************************

%\cvitem[<Option>]{<Argument 1>}{<Argument 2>}
%\cvdoubleitem[<Option>]{Argument 1}{Argument 2}{Argument 3}{Argument 4}
%\cvlistitem[<Label>]{<Item>}
%\cvlistdoubleitem[<Label>]{<Item 1>}{<Item 2>}
%\cventry[<Option>]{<Argument 1>}{<Argument 2>}{<Argument 3>}{<Arg. 4>}{<Arg. 5>}{<Arg. 6>}
%\cvitemwithcomment[<Option>]{<Argument 1>}{<Argument 2>}{<Argument 3>}





%Ergänzung:
%Wenn kein Photo angegeben wird, dann setze irgend ein Bild aus den Logos ein und setze die Größe auf Null!
\providecommand{\HTWmakecvtitle}{
\makeatletter
\ifx\@photo\empty
\setlength{\fboxsep}{0pt}
\photo[0cm]{pspictures/figures/logo/logo_long}
\fi
\makeatother
\makecvtitle
}

\HTWmakecvtitle					% CV Titel ausgeben


\makeatletter
\section{\textsc{Persönliche Angaben}}
\cventry{Name}{\@firstname\hspace{0em} \@lastname}{}{}{}{}
\cventry{Anschrift}{\@addressstreet , \@addresscity}{}{}{}{}
\cventry{Geboren}{08.08. 1995}{Berlin}{}{}{}
\cventry{Familienstand}{ledig}{}{}{}{}
\cventry{Nationalität}{deutsch}{}{}{}{}





\section{\textsc{Kontakt}}
\cventry{Telefon}{\@mobile}{}{}{}{}
\cventry{E-Mail}{\@email}{}{}{}{}








\section{\textsc{Werdegang}}
\subsection{\textsc{Beruflich}}
\cventry{seit 04.2017}{ABC Programmierer GmbH}{}{}{}{Software-Entwicklung (freiberuflich),\newline Programmiersprachen: C und Python.}









\subsection{\textsc{Akademisch}}
\cventry{10.2016 bis 04.2017}{Hochschule für Technik und Wirtschaft Berlin Berlin}{}{}{}{}









\subsection{\textsc{Schulisch}}
\cventry{09.2015 bis 07.2016}{Gymnasium XY}{}{}{}{}









\subsection{\textsc{Sprachkenntnisse}}
\cvlanguage{Deutsch}{Muttersprache.}{}
\cvlanguage{Englisch}{Verhandlungssicher.}{}









\section{\textbf{Text der Funktionsbefehle}}
\subsection{Text des Funktionsbefehles 1}
\cvitem{Argument 1}{Argument 2}
\cvitem{Argument 3}{Argument 4}




\subsection{Text des Funktionsbefehles 2}
\cvdoubleitem{Argument 1}{Argument 2}{Argument 3}{Argument 4}
\cvdoubleitem{Argument 5}{Argument 6}{Argument 7}{Argument 8}




\subsection{Text des Funktionsbefehles 3}
\cvlistitem{Eintrag 1}
\cvlistitem{Eintrag 2}
\cvlistitem{Eintrag 3}




\subsection{Text des Funktionsbefehles 4}
\cvlistdoubleitem{Eintrag 1.1}{Eintrag 1.2}
\cvlistdoubleitem{Eintrag 2.1}{Eintrag 2.2}
\cvlistdoubleitem{Eintrag 3.1}{Eintrag 3.2}




\subsection{Text des Funktionsbefehles 5}
\cventry{Eintrag 11}{Eintrag 12}{Eintrag 13}{Eintrag 14}{Eintrag 15}{Eintrag 16}
\cventry{Eintrag 21}{Eintrag 22}{Eintrag 23}{Eintrag 24}{Eintrag 25}{Eintrag 26}




\subsection{Text des Funktionsbefehles 5}
\cvitemwithcomment{Eintrag 11}{Eintrag 12}{Eintrag 13}
\cvitemwithcomment{Eintrag 21}{Eintrag 22}{Eintrag 23}




\subsection{Text des Funktionsbefehles 7}
\cvlanguage{Deutsch}{Muttersprache.}{A1}








\vspace{3cm}
\quad\includegraphics[scale=\signatureScale]{\signature}
%\quad\includegraphics[scale=0.5]{fig/SignaturPlatzhalter.png}
\vskip -1.8cm
Berlin, \today


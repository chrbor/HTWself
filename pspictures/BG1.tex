\documentclass[11pt]{standalone}
%Präambel:

\usepackage[T1]{fontenc}
\usepackage[utf8]{inputenc}
\usepackage[ngerman]{babel}


\usepackage{pstricks}
\usepackage{xcolor}
\usepackage{HTWColor}

%Ermöglicht das Einbinden von PNG und JPEG bei pstricks:
\usepackage{auto-pst-pdf}

%Definiere Unit-Größe:
\psset{xunit=0.3cm,yunit=0.3cm}%Bei a4=210mm x 297mm ergeben sich 70 x 99 Felder

%Logik:
\usepackage{multido}

%\usepackage{HTWMath}

\begin{document}
%18-60
\begin{pspicture}[showgrid=false](0,0)(70,99)
% \rput(52,89){\includegraphics[height=1.5cm, keepaspectratio=true]{figures/logo/logo_long}}
% \rput(10,89){\includegraphics[height=4cm, keepaspectratio=true]{figures/logo/CE_logo}}
% \rput[l](15,90){\texttt{\bfseries\large Fachbereich 1}} 
% \rput[l](15,88){\texttt{\bfseries\large Computer Engineering (Master)}} 
 
% \rput(18,89){\includegraphics[height=1.5cm, keepaspectratio=true]{figures/logo/logo_long}}
% \rput(60,89){\includegraphics[height=4cm, keepaspectratio=true]{figures/logo/CE_logo}}
% \rput[r](55,90){\texttt{\bfseries\large Fachbereich 1}} 
% \rput[r](55,88){\texttt{\bfseries\large Computer Engineering (Master)}}  
 
 \psframe[
 framearc=0,
 linewidth=3pt,
 linecolor=HTWGreen100
 ](5,5)(65,94)
\end{pspicture}

%\def\x{13}
%\FPinc{\x}
%Der wert x liegt bei \x .\\

\end{document}
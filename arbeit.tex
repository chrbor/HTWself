%===========================================%
% Optionen: cv/exam                         %
% exam braucht aufgrund der Graphiken ein   % 
% bisschen länger zum kompilieren           %
%                                           %
% Hinweis:                                  %
% die Dokumentation der Klasse kann         %
% entweder extern oder aber durch den       %
% befehl \h geöffnet werden.                %
% Spezielle Befehle können mit dem Befehl   %
% \help{<befehl>} nachgeschlagen werden.    %
% (bitte <befehl> ohne Backslash eingeben)  %
%                                           %
% Weiterhin benutzt die Option exam das     %
% Paket "currfile", dass den Pfad des       %
% Projektes aus dem recorder-file entnimmt  %
% Daher bitte in den Kompilier-Optionen     %
% bei xelatex -recorder einfügen.           %
%                                           %
%===========================================%
% Viel Spaß mit der Klasse HTWself ;)       %
%===========================================%

\documentclass[exam]{HTWself}

\usepackage{HTWCode}

\begin{document}
%\help{cv}
%\help{HTWCV}
%\h

\h

\HTWVBracketbox{1}{
\lstinputlisting[
style=C-editor, 
basicstyle=\footnotesize ,
lastline=43]{code/main.cpp}
}{main.m}

\HTWVBracketbox{1}{
\lstinputlisting[
style=C-editor, 
basicstyle=\footnotesize ,
firstline=44,
firstnumber=44,
lastline=87]{code/main.cpp}
}{main.m}

\HTWFramebox{1}{
\lstinputlisting[
style=C-editor, 
basicstyle=\footnotesize ,
firstline=88,
firstnumber=88,
lastline=131]{code/main.cpp}
}
\end{document}

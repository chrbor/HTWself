%===========================================%
% Optionen: cv/exam                         %
% exam braucht aufgrund der Graphiken ein   % 
% bisschen länger zum kompilieren           %
%                                           %
% Hinweis:                                  %
% die Dokumentation der Klasse kann         %
% entweder extern oder aber durch den       %
% befehl \h geöffnet werden.                %
% Spezielle Befehle können mit dem Befehl   %
% \help{<befehl>} nachgeschlagen werden.    %
% (bitte <befehl> ohne Backslash eingeben)  %
%                                           %
% Weiterhin benutzt die Option exam das     %
% Paket "currfile", dass den Pfad des       %
% Projektes aus dem recorder-file entnimmt  %
% Daher bitte in den Kompilier-Optionen     %
% bei xelatex -recorder einfügen.           %
%                                           %
%===========================================%
% Viel Spaß mit der Klasse HTWself ;)       %
%===========================================%

\documentclass[cv]{HTWself}

\begin{document}
%\help{cv}
%\help{HTWCV}
%\h

\ifthenelse{\equal{\type}{exam}}{%über type kann die option abgerufen werden

	\HTWTitle{Abschlussarbeit}
	\HTWSubtitle{Implementierung von Videodenoising-Algorithmen zum Entrauschen von faseroptischen Signalen}
	\HTWReference{DOC\_01284-4C}	

	\HTWStudent{Christopher}{Borchadt}{s0549085}{cborchardt95@gmail.com}
	%"erfundene" Studenten zum Zwecke der Demo:
	\HTWStudent{Leon}{Walsinger}{s0532124}{}%leere Eingabe bewirkt, dass aus der Matrikelnummer die Adresse generiert wird
	\HTWStudent{Susanne}{Habrecht}{s0511111}{/}%einzelner Buchstabe bewirkt, dass die Adresse weggelassen wird
	\HTWStudent{Peter}{Hausmann}{}{Hausmann@gmail.com}%Es können auch Leute ohne Matrikelnummer eingefügt werden	

	\HTWCourse{\LaTeX\hspace{0em} für Fortgeschrittene}
	\HTWReviewer{Abdullah Bassem Peter Al Jamal}
	
	\HTWCover
	\HTWListsOfx
	\chapter{Das Schachbrett}
	Folgend das Schachbrett:\\
	\psset{xunit=1cm, yunit=1cm}
	\begin{center}
		\HTWChecker{red}{blue}{4}{4}{}		
	\end{center}
	\vskip10mm
	Das Schachbrett lässt sich nahzu beliebig skalieren:\\
		\begin{center}
		\HTWChecker{HTWGreen100}{HTWBlue100}{12}{3}{}		
	\end{center}
	
	\newpage
	Es kann sogar eine Skalierung in der Art eines Schachfeldes hinzugefügt werden:\\
		\begin{center}
		\HTWChecker{black}{black}{8}{8}{x}
		\end{center}
	\par
		
	\chapter{Überschriften}
	\section{Sektion}
	\subsection{Untersektion}
	\subsubsection{Unteruntersektion}
	Ich bin ein normaler Text
	\newpage
	
	\chapter{Listen und Boxen}
	\section{Startzüge}
	\blindtext
	\begin{HTWBulletList}
	\item
	Mache den ersten Zug\\Der Spieler, der startet, hat immer einen leichten Vorteil
	\item
	Versuche die Dominanz des Feldes zu ergreifen
	\item
	Schütze deine Figuren mit anderen Figuren
	\end{HTWBulletList}
	\blindtext	
	
	\newpage
	\subsection{Varianten}
	\blindtext
	
	\begin{HTWdefList}
	\item[Die erste Sache]
	Mache den ersten Zug\\Der Spieler, der startet hat immer einen leichten Vorteil
	\item[Die zweite Sache]
	Versuche die Dominanz des Feldes zu ergreifen
	\item[Die dritte Sache]
	Schütze deine Figuren mit anderen Figuren
	\end{HTWdefList}
	%\newpage
	
	\begin{wrapfigure}{r}{0.5\textwidth}%[l]eft, [r]ight, [i]nner, [o]uter
	\HTWSpeechbubblebox{0.75}{
	Sollten Sie mal eine interessante Information mit ihren Lesern teilen wollen, dann können sie diese Sprechblase nutzen!  
	}{Info}{}
	\end{wrapfigure}
	\blindtext
	\\\par
	\blindtext
	\newpage
	\HTWQuotebox{1}{\blindtext}{Test}
	\blindtext
	\HTWInfobox{1}{\blindtext}{Info}
	\blindtext
	\newpage
	
	\HTWBracketbox{1}{
	{\bfseries\large\color{HTWGreen100} Wichtige Sachen zum Schreiben}\\\\
	Der Satz des Pythagoras lautet:\\
	\begin{equation}
	a^2 + b^2 = c^2
	\end{equation}
	Wenn man c haben möchte, dann muss man die Gleichung umstellen:\\
	\begin{equation}
	c = \sqrt{a^2 + b^2}
	\end{equation}
	}
	\blindtext
	\blindtext
	
	\HTWVBracketbox{1}{\blindtext}{Code1.code}
	\newpage	
	\blindtext	
	\HTWFramebox{1}{\blindtext}
	\blindtext
	\newpage
	
	\HTWLastpage
}{
%\h
	\HTWTitle{Bewerbung}
	\HTWSubtitle{Um einen Ausbildungsplatz als}
	\HTWJobname{Nachhilfelehrkraft (m/w)}
%	\photo[0pt]{ModernCV/fig/photo.jpg}%0pt Nimmt das Bild aus den Lebenslauf raus
	\photo{}%0 Kein Bild(nicht empfohlen, da sonst ein schlechterer Eindruck entsteht)
	
	\HTWAttachment{Lebenslauf}{}{}%Lebenslauf wird von der Klasse selbst erstellt, daher wird keine externe Datei inkludiert
	\HTWAttachment{Background.pdf}{pspictures/BG1.pdf}{pdf}	
	\HTWAttachment{Anschreiben}{ModernCV/parts/Anschreiben.tex}{tex}
	\HTWAttachment{Tolles Bild}{ModernCV/fig/photo.jpg}{jpg}

%	\HTWAttachmentAngle{-90}
	\HTWAttachmentHeight{0.7}%passe die Höhe der jpg/png-Dateien an
	\HTWCover	
	\HTWCV
	\HTWIncludeAttachments
%	\HTWIncludeAttachment{Tolles Bild}
	\HTWIncludeAttachment{nichtGegeben}%Attachments, die nicht definiert sind, werden übersprungen
%	\HTWLastpage
}
\end{document}
